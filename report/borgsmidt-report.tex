%!TEX TS-program = xelatex
%!TEX encoding = UTF-8 Unicode

\documentclass[
paper=a4,
oneside,
fontsize=11pt,
numbers=noenddot,
headinclude=false, % Count header as margin
footinclude=false, % Count footer as margin
%mpinclude=true, % Include this to leave space for margin notes
fleqn,             % Equations on left instead of center
DIV=8
]{scrartcl}

\usepackage[automark]{scrpage2}
\makeatletter\@ifclassloaded{scrreprt}{%
  \pagestyle{scrheadings}%
  \setlength{\marginparwidth}{7em}%
  \setlength{\marginparsep}{2em}%
}\makeatother

\addtolength{\textheight}{35mm}

\usepackage[usenames,dvipsnames]{xcolor} %use color with the command \textcolor{spot}{some text to
%with your color previously defined}
\definecolor{seccol}{rgb}{0.6,0,0}
\definecolor{spot}{rgb}{0.6,0,0}
\definecolor{boxfill}{rgb}{.96,.97,1}
\definecolor{halfgray}{gray}{0.55}
\definecolor{webgreen}{rgb}{0,.5,0}
\definecolor{royalblue}{cmyk}{0.8, 0.50, 0, 0}

% Graffiti as in GKP's book "Concrete Mathematics"
% thanks to Lorenzo Pantieri and Enrico Gregorio
\def\graffito@setup{%
   \textmpar\footnotesize\setstretch{1}%
   \parindent=0pt \lineskip=0pt \lineskiplimit=0pt %
   \tolerance=2000 \hyphenpenalty=300 \exhyphenpenalty=300%
   \doublehyphendemerits=100000%
   \finalhyphendemerits=\doublehyphendemerits}
%\DeclareRobustCommand{\graffito}[1]{\marginpar%
% [\graffito@setup\raggedleft\hspace{0pt}{#1}]%
% {\graffito@setup\raggedright\hspace{0pt}{#1}}}
\let\oldmarginpar\marginpar
\renewcommand{\marginpar}[1]{\oldmarginpar%
 [\graffito@setup\raggedleft\hspace{0pt}{#1}]%
 {\graffito@setup\raggedright\hspace{0pt}{#1}}}

\usepackage{polyglossia}
\setmainlanguage[variant=american]{english}
\usepackage{eukdate}   % International date format - Day Month Year

\usepackage{xspace} % Helps eliminate unwanted whitespace after TeX commands
\usepackage[numbers]{natbib}
\setcitestyle{aysep={}} % No comma between author and year
\usepackage[nottoc]{tocbibind}
 
\usepackage{url}
\usepackage{listings}
\usepackage{longtable}
\usepackage{booktabs}
\usepackage{tabularx} % better tables
\usepackage{textcomp}  % additional characters
\usepackage{braket}
\usepackage{amssymb}
\usepackage{amsmath}
\usepackage{mathtools}
\usepackage{graphicx}  % to include graphics
\usepackage{wrapfig}   % to get nice wrap of figures
\usepackage{placeins} % For \FloatBarrier to stop floats
\usepackage{titlesec}
\usepackage{textcase} % for \MakeTextUppercase 
\usepackage{mparhack} % get marginpar right
\usepackage{multicol}
\usepackage{xltxtra}
\usepackage{microtype}
\usepackage{setspace}
\setstretch{1.12} % Additional space between lines
\usepackage{relsize}
\usepackage[printonlyused,smaller]{acronym}
\newcommand{\acrofont}[1]{\textrm{\textmd{#1}}}
\renewcommand*{\acsfont}[1]{\textsmaller{\acrofont{#1}}}

% FONTSs & SECTION LAYOUT

\setmainfont[Scale=1.05, Mapping=tex-text, Numbers={Lining, Monospaced}]{Arno Pro}
\setmonofont[Scale=MatchLowercase]{Menlo}
\newfontfamily{\textmpar}[Mapping=tex-text, Numbers={OldStyle, Proportional}]{Arno Pro Italic SmText}
\newfontfamily{\regnum}[Mapping=tex-text, Numbers={Lining, Monospaced}]{Arno Pro}

% Must load eulervm after loading fonts
\usepackage[small]{eulervm}   % Use beautiful Euler maths (with main font number glyphs)

% Reset main font to old-style number after loading eulervm
\setmainfont[Scale=1.1, Mapping=tex-text, Numbers={OldStyle, Proportional}]{Arno Pro}


% Special characters
\newcommand{\oldamp}{\begingroup\addfontfeature{Alternate=2}\&\endgroup}

% Letterspacing
\usepackage{soul}
\sodef\allcapsspacing{\upshape}{0.15em}{0.65em}{0.6em}%
\sodef\lowsmallcapsspacing{\scshape}{0.075em}{0.5em}{0.6em}%
\DeclareRobustCommand{\spacedallcaps}[1]{\MakeTextUppercase{\allcapsspacing{#1}}}%   
\DeclareRobustCommand{\spacedlowsmallcaps}[1]{\MakeTextLowercase{\textsc{\lowsmallcapsspacing{#1}}}}%

% Table commands
\newcommand{\tableheadline}[1]{\multicolumn{1}{c}{\spacedlowsmallcaps{#1}}}

% ********************************************************************                
% headlines
% ********************************************************************
\makeatletter\@ifclassloaded{scrreprt}{%
\clearscrheadings
\setheadsepline{0pt}
\renewcommand{\sectionmark}[1]{\markright{\thesection\enspace\spacedlowsmallcaps{#1}}} 
\lehead{\mbox{\llap{\small\thepage\kern2em}\headmark\hfil}}
\rohead{\mbox{\hfil{\headmark}\rlap{\small\kern2em\thepage}}}
\renewcommand{\headfont}{\small}
}\makeatother

\makeatletter\@ifclassloaded{scrartcl}{%
  \def\toc@heading{%
    \section*{\contentsname}%
    \@mkboth{\spacedlowsmallcaps{\contentsname}}{\spacedlowsmallcaps{\contentsname}}}
}\makeatother
  
% %*********************************************************
% Sectioning
% %*********************************************************

\makeatletter\@ifclassloaded{scrartcl}{%
% Article sections \FloatBarrier
  \titleformat{\section}
    {\relax}{\large\textsc{\MakeTextLowercase{\thesection}}}{1em}{\large\spacedlowsmallcaps}
% subsections
\titleformat{\subsection}
  {\relax}{\textsc{\MakeTextLowercase{\thesubsection}}}{1em}{\normalsize\itshape}        
}\makeatother

% subsubsections
\titleformat{\subsubsection}
  {\relax}{\textsc{\MakeTextLowercase{\thesubsubsection}}}{1em}{\normalsize\itshape}        
% paragraphs
\titleformat{\paragraph}[runin]
  {\normalfont\normalsize}{\theparagraph}{0pt}{\spacedlowsmallcaps}    
% descriptionlabels
\renewcommand{\descriptionlabel}[1]{\hspace*{\labelsep}\spacedlowsmallcaps{#1}}

\titlespacing*{\section}{0pt}{1.25\baselineskip}{1\baselineskip} 
\titlespacing*{\subsection}{0pt}{1.25\baselineskip}{1\baselineskip}
\titlespacing*{\paragraph}{0pt}{1\baselineskip}{1\baselineskip}

% ********************************************************************
% footnotes setup   
% ********************************************************************
%\RequirePackage{footmisc}  % [bottom] norule para symbol* marginal perpage
    % KOMA-command, footnotemark not superscripted at the bottom
    \deffootnote{0em}{0em}{\thefootnotemark\hspace*{.5em}}      

% ********************************************************************                
% layout of the TOC, LOF and LOT (LOL-workaround see next section)
% ********************************************************************
\usepackage[titles]{tocloft}
% avoid page numbers being right-aligned in fixed-size box              
\newlength{\newnumberwidth}
\settowidth{\newnumberwidth}{999} % yields overfull hbox warnings for pages > 999
\cftsetpnumwidth{\newnumberwidth}

\newcounter{dummy} % necessary for correct hyperlinks (to index, bib, etc.)
    
% have the bib neatly positioned after the rest
\newlength{\beforebibskip}  
\setlength{\beforebibskip}{0em}
    
% space for more than nine chapters
\newlength{\newchnumberwidth}
\settowidth{\newchnumberwidth}{.} % <--- tweak here if more space required
%  \addtolength{\cftchapnumwidth}{\newchnumberwidth}%
  \addtolength{\cftsecnumwidth}{\newchnumberwidth}
  \addtolength{\cftsecindent}{\newchnumberwidth}
  \addtolength{\cftsubsecnumwidth}{\newchnumberwidth}
  \addtolength{\cftsubsecindent}{2\newchnumberwidth}
  \addtolength{\cftsubsubsecnumwidth}{\newchnumberwidth}
	  
  % % chapters
  % \renewcommand{\cftchappresnum}{\scshape\MakeTextLowercase}%
  % \renewcommand{\cftchapfont}{\scshape}%
  % \renewcommand{\cftchappagefont}{\normalfont}%
  % \renewcommand{\cftchapleader}{\hspace{1.5em}}% 
  % \renewcommand{\cftchapafterpnum}{\cftparfillskip}% 

  % sections
  \renewcommand{\cftsecpresnum}{\small\scshape\MakeTextLowercase}%
  \renewcommand{\cftsecfont}{\small\scshape\MakeTextLowercase}%
  \renewcommand{\cftsecpagefont}{\normalfont}%
  \renewcommand{\cftsecleader}{\hspace{1.5em}}% 
  \renewcommand{\cftsecafterpnum}{\cftparfillskip}%
  %\ifthenelse{\boolean{@tocaligned}}{\renewcommand{\cftsecindent}{0em}}{\relax}

  % subsections
  \renewcommand{\cftsubsecpresnum}{\scshape\MakeTextLowercase}%
  \renewcommand{\cftsubsecfont}{\normalfont}%
  \renewcommand{\cftsubsecleader}{\hspace{1.5em}}% 
  \renewcommand{\cftsubsecafterpnum}{\cftparfillskip}%   
  %\ifthenelse{\boolean{@tocaligned}}{\renewcommand{\cftsubsecindent}{0em}}{\relax}

  % subsubsections
  \renewcommand{\cftsubsubsecpresnum}{\scshape\MakeTextLowercase}%
  \renewcommand{\cftsubsubsecfont}{\normalfont}%
  \renewcommand{\cftsubsubsecleader}{\hspace{1.5em}}% 
  \renewcommand{\cftsubsubsecafterpnum}{\cftparfillskip}%   
  %\ifthenelse{\boolean{@tocaligned}}{\renewcommand{\cftsubsubsecindent}{0em}}{\relax}

  % figures     
  \renewcommand{\cftfigpresnum}{\scshape\MakeTextLowercase}% 
  \renewcommand{\cftfigfont}{\normalfont}%                 
  \renewcommand{\cftfigleader}{\hspace{1.5em}}% 
  \renewcommand{\cftfigafterpnum}{\cftparfillskip}%
  \renewcommand{\cftfigpresnum}{}% Don't prefix with 'Figure'
  \newlength{\figurelabelwidth}
  \settowidth{\figurelabelwidth}{\cftfigpresnum~9.9}
  \addtolength{\figurelabelwidth}{1.2em}
  \cftsetindents{figure}{0em}{\figurelabelwidth}
  \gappto\captionsenglish{\renewcommand{\listfigurename}{Figures}}

  % tables
  \renewcommand{\cfttabpresnum}{\scshape\MakeTextLowercase}%
  \renewcommand{\cfttabfont}{\normalfont}%
  \renewcommand{\cfttableader}{\hspace{1.5em}}% 
  \renewcommand{\cfttabafterpnum}{\cftparfillskip}%   
  \renewcommand{\cfttabpresnum}{}% Don't prefix with 'Table'
  %\newlength{\tablelabelwidth}
  %\settowidth{\tablelabelwidth}{\cfttabpresnum~99}
  %\addtolength{\tablelabelwidth}{2.5em}
  % \cftsetindents{table}{0em}{\tablelabelwidth}
  \cftsetindents{table}{0em}{\figurelabelwidth}
%  \gappto\captionsenglish{\renewcommand{\tablename}{Definition}}
%  \gappto\captionsenglish{\renewcommand{\listtablename}{Definitions}}

  % listings
  \newlistof{listings}{lol}{\lstlistlistingname}%
  \renewcommand{\cftlistingspresnum}{\scshape\MakeTextLowercase}%
  \renewcommand{\cftlistingsfont}{\normalfont}%
  \renewcommand{\cftlistingspresnum}{}% Don't prefix with 'Listing'
  \renewcommand{\cftlistingspagefont}{\normalfont}%
  \renewcommand{\cftlistingsleader}{\hspace{1.5em}}%
  \renewcommand{\cftlistingsafterpnum}{\cftparfillskip}%
  %\newlength{\listingslabelwidth}%
  %\settowidth{\listingslabelwidth}{\cftlistingspresnum~99}%
  %\addtolength{\listingslabelwidth}{}%
  % \cftsetindents{listings}{0em}{\listingslabelwidth}%
  \cftsetindents{listings}{0em}{\figurelabelwidth}
  \makeatletter%
  \let\l@lstlisting\l@listings%
  \let\lstlistoflistings\listoflistings%
  \makeatother
  % \renewcommand\lstlistingname{Example}
  % \renewcommand\lstlistlistingname{Examples}
  % \def\lstlistingautorefname{Example}

  % dirty work-around to get the spacing after the toc/lot/lof-titles right    
  \AtBeginDocument{\addtocontents{toc}{\protect\vspace{-1em}}}

  \newcommand{\tocEntry}[1]{% for bib, etc.
    \spacedlowsmallcaps{#1}
  }

% %    % remove the vertical space between lof/lot entries of different chapters
% 		\ifthenelse{\boolean{@listsseparated}}{%
% 		 \PackageWarningNoLine{classicthesis}{Option "listsseparated" deprecated as of version 2.9.}%
% 		}{\relax}
% %    \ifthenelse{\boolean{@listsseparated}}{%
% %        \AtBeginDocument{%
% %            \addtocontents{lof}{\protect\vspace{-\cftbeforechapskip}}%
% %            \addtocontents{lot}{\protect\vspace{-\cftbeforechapskip}}%
% %            \ifthenelse{\boolean{@listings}}%        
% %    				{%
% %             	\addtocontents{lol}{\protect\vspace{-\cftbeforechapskip}}%
% %            }{\relax}%
% %        }%
% %    }{%
%         \DeclareRobustCommand*{\deactivateaddvspace}{\let\addvspace\@gobble}% 
%         \AtBeginDocument{%      
%             \addtocontents{lof}{\deactivateaddvspace}% 
%             \addtocontents{lot}{\deactivateaddvspace}%
%     				\ifthenelse{\boolean{@listings}}% 
%     				{%
%              	\addtocontents{lol}{\deactivateaddvspace}%
%             }{\relax}%                  
%         }%
% %    } 
   
% ********************************************************************
% Hyperreferences setup
%*******************************************************
\usepackage{hyperref}
\hypersetup{%
    colorlinks=true, linktocpage=true, pdfstartpage=3, pdfstartview=FitV,%
    % uncomment the following line if you want to have black links (e.g., for printing)
    %colorlinks=false, linktocpage=false, pdfborder={0 0 0}, pdfstartpage=3, pdfstartview=FitV,% 
    breaklinks=true, pdfpagemode=UseNone, pageanchor=true, pdfpagemode=UseOutlines,%
    plainpages=false, bookmarksnumbered, bookmarksopen=true, bookmarksopenlevel=1,%
    hypertexnames=true, pdfhighlight=/O,%hyperfootnotes=true,%nesting=true,%frenchlinks,%
    urlcolor=RoyalBlue, linkcolor=spot, citecolor=spot, %pagecolor=RoyalBlue,%
    %urlcolor=Black, linkcolor=Black, citecolor=Black, %pagecolor=Black,%
    pdftitle={Advanced Programming 2013 Exam Report},%the title
    pdfauthor={Rasmus Borgsmidt},%your name
    pdfsubject={},%
    pdfkeywords={},%
    pdfcreator={XeLaTeX},%
    pdfproducer={XeLaTeX}%
}

% ******************************************************
% Listings setup
%*******************************************************
\lstset{
  basicstyle=\ttfamily\small,
  commentstyle=\color{webgreen}\ttfamily,
  numbers=left,
  numberstyle=\ttfamily\scriptsize,
  numbersep=8pt,
  showstringspaces=false,
  breaklines=true,
  backgroundcolor=\color{boxfill},
  frame=tb,
  framerule=\heavyrulewidth, % defined in the booktabs package
  belowcaptionskip=.75\baselineskip
}

% Get single spacing in listings
\let\oldlst\lstlisting
\def\lstlisting{\par\setstretch{0.8}\oldlst}
\let\oldlstin\lstinputlisting
\def\lstinputlisting{\par\setstretch{0.8}\oldlstin}

% Use regular - instead of dash in listings
\makeatletter
\lst@CCPutMacro\lst@ProcessOther {"2D}{\lst@ttfamily{-{}}{-{}}}
\@empty\z@\@empty
\makeatother


% Disable single lines at the start of a paragraph (Schusterjungen)
\clubpenalty = 10000
% Disable single lines at the end of a paragraph (Hurenkinder)
\widowpenalty = 10000 
\displaywidowpenalty = 10000 % formulas


\begin{document}

\title{\textcolor{spot}{\rmfamily\mdseries\spacedlowsmallcaps{Exam Report}}}
\subtitle{\rmfamily\mdseries\itshape\normalsize{Advanced Programming 2013}}
\author{\large Rasmus Borgsmidt}
\date{}
\vspace{-30pt}\maketitle
\pdfbookmark[1]{\contentsname}{tableofcontents}
\setcounter{tocdepth}{2} % <-- 2 includes up to subsections in the ToC
\manualmark
\vspace{-30pt}\tableofcontents
%\vspace{30pt}

\addcontentsline{toc}{section}{\lstlistlistingname}
\lstlistoflistings

%\addcontentsline{toc}{section}{Tables}
\listoftables

\fontdimen2\font=1.25\fontdimen2\font% interword space
\fontdimen3\font=1.25\fontdimen3\font% interword stretch
\fontdimen4\font=\fontdimen4\font% interword shrink

\section*{Introduction}
\addcontentsline{toc}{section}{Introduction}

\section{Salsa Language Parser}

\subsection{Choice of Parser Library}

\subsection{Grammar Transformations}

Goal: to make the grammar {\regnum LL(1)}.

Emphasis in correctness. Code can be improved for efficiency later. Fx.\ using
many1 etc.

Assumptions: || is left-associative. The original grammar does not specify this,
This was chosen because of symmetri with {\tt '@'}.

{\tt '+'} and {\tt '-'} have the same precedence for expressions

As outlined by \citet[p.~69]{mogensen2011}.

\begin{enumerate}
\item Eliminate ambiguity
\item Eliminate left-recursion
\item Perform left-factorization where required
\end{enumerate}

Note: And it is only required when several productions {\em for the same
  nonterminal} begin with the same sequence of symbols; for example, it is not a
problem that both {\em Pos} and {\em Prim} have a production that begins with
{\tt '('}, as long as they have only one each.

The resulting grammar makes it explicit that {\tt '@'} has higher precedence
than {\tt '||'}. In other words, it makes it impossible to derive the wrong
parse tree from a given input.

\begin{table}[h]
  \centering\regnum
  \caption{{\regnum LL(1)} {\scshape Salsa} Grammar}\label{grammar}
  \begin{tabularx}{\textwidth}{rcX} \toprule
    {\em Program}    & $::=$ & {\em DefComs}\\
    {\em DefComs}    & $::=$ & {\em DefCom} {\em DefComs$_{*}$}\\
    {\em DefComs$_{*}$}   & $::=$ & {\em DefCom} {\em DefComs$_{*}$}\\
               & $|$   & $\epsilon$\\
    {\em DefCom}     & $::=$ & {\em Command}\\
               & $|$   & {\em Definition}\\
    {\em Definition} & $::=$ & {\tt 'viewdef'} {\em VIdent} {\em Expr} {\em Expr}\\
               & $|$   & {\tt 'rectangle'} {\em SIdent} {\em Expr} {\em Expr} {\em Expr} {\em Expr} {\em Colour}\\
               & $|$   & {\tt 'circle'} {\em SIdent} {\em Expr} {\em Expr} {\em Expr} {\em Colour}\\
               & $|$   & {\tt 'view'} {\em VIdent}\\
               & $|$   & {\tt 'group'} {\em VIdent} {\tt '['} {\em VIdents} {\tt ']'}\\
    {\em Command}    & $::=$ & {\em Command1} {\em Command$_{*}$}\\
    {\em Command$_{*}$}    & $::=$ & {\tt '||'} {\em Command1} {\em Command$_{*}$}\\
               & $|$ & $\epsilon$ \\
    {\em Command1}    & $::=$ & {\em Command2} {\em Command1$_{*}$}\\
    {\em Command1$_{*}$}    & $::=$ & {\tt '@'} {\em VIdent} {\em Command1$_{*}$}\\
               & $|$ & $\epsilon$ \\
    {\em Command2}    & $::=$ & {\em SIdents} {\tt '->'} {\em Pos}\\
               & $|$   & {\tt '\{'} {\em Command} {\tt '\}'}\\

    {\em VIdents}    & $::=$ & {\em VIdent} {\em VIdents$_{*}$}\\
    {\em VIdents$_{*}$}    & $::=$ & {\em VIdent} {\em VIdents$_{*}$}\\
               & $|$   & $\epsilon$\\
    {\em SIdents}    & $::=$ & {\em SIdent} {\em SIdents$_{*}$}\\
    {\em SIdents$_{*}$}    & $::=$ & {\em SIdent} {\em SIdents$_{*}$}\\
               & $|$   & $\epsilon$\\
    {\em Pos}        & $::=$ & {\tt '('} {\em Expr} {\tt ','} {\em Expr} {\tt ')'}\\
               & $|$   & {\tt '+'} {\tt '('} {\em Expr} {\tt ','} {\em Expr} {\tt ')'}\\
    {\em Expr}       & $::=$ & {\em Prim} {\em Expr$_{*}$}\\
    {\em Expr$_{*}$}       & $::=$ & {\tt '+'} {\em Prim} {\em Expr$_{*}$}\\
               & $|$   & {\tt '-'} {\em Prim} {\em Expr$_{*}$}\\
               & $|$   & $\epsilon$\\
    {\em Prim}       & $::=$ & {\em integer}\\
               & $|$   & {\em SIdent} {\tt '.'} {\em Coord}\\
               & $|$   & {\tt '('} {\em Expr} {\tt ')'}\\
    {\em Coord}       & $::=$ & {\tt 'x'} $|$ {\tt 'y'}\\
    {\em Colour}     & $::=$ & {\tt 'blue'} $|$ {\tt 'plum'} $|$ {\tt 'red'} $|$ {\tt 'green'} $|$ {\tt 'orange'}\\
    \bottomrule
  \end{tabularx}
\end{table}

\subsection{Testing}
\label{sec:testing}

Do not want to expose functions from module in order to test. But still come up
with strategy for somewhat orthogonal testing.

\subsection{Assessment}
\label{sec:assessment}

I believe this is a complete and correct solution to question 1.

Allow trailing whitespace
SIdents and VIdents parsers could be parameterized and combined.

Comment on that it is particularly tricky to do identifiers separated by space.

extensive test suite
machine-like translation from grammar
well-chosen grammar transformations

\section{Salsa Language Interpreter}

A context comprises an environment and a state; the environment specifies views,
shapes and other definitions, the state specifies the position of each shape on
the views. In that sense, a given state specifies a {\em key frame} and the
transition from each key frame to the next is specified by the difference
between one state and the next.

Apart from possible implications on ordering of graphics instructions, will it
impact the resulting Animation if our interpreter executes concurrent commands
in parallel or sequentially?

I don't think so, given the 

If views are defined later in the animation, they just stay blank until
something happens on them.

Comment on the point of the separate SalsaCommand monad, perhaps to make it
clear that commands cannot impact the environment.

SalsaCommand is responsible for generating frames -- why else would the frame
rate be required in the context?

Explain why this is the case (concurrent execution - other reasons?):
{- NOTE: Two sequenced SalsaCommands are deliberately executed in the same context,
 - the state is not updated before the second execution -}

Unnecessary to nest environment inside context -- as long as commands can only
amend state.

In a blank context, there should be a blank frame from the outset. Explain how
the frames (and frame sets) are manipulated.  Why are we using frame sets?

I feel that adding a new shape onto the frame should also be a SalsaCommand
(explain why it is not)

Talk about not using more monads -- eval is too simple

-- When talking about this type, point out that no error handling is necessary
-- This type reflects that running a command cannot update the environment,
-- just the state
-- The type captures the effect of a command in a given context
newtype SalsaCommand a = SalsaCommand { runSC :: Context -> (a, State) }

\section{Atomic Transaction Server}

Why have I chosen to use OTP?
Talk about time outs and client-provided query functions

Using a waiting list of clients is safe because query\_t is blocking so each
client can appear at most once in the list

clients can build a timeout into the query and update functions

Copy in text from test file

theoretic problem in choiceUpdate with many processes if choiceUpdate is still
creating new processes while the first has already completed.

From the exam text:

"[...] the result of choiceUpdate is the result of this commit."

In a strict sense, the "result of this commit" 

The direct return value of 

\section{Conclusion}


\bibliographystyle{plainnat}
\bibliography{borgsmidt-report}

\appendix

\newcommand{\shcodefile}[2]{%
\lstinputlisting[basicstyle=\ttfamily\footnotesize, language={},%
  caption={\tt #1}, label=lst:#2]{../#1}}

\clearpage
\section{Question 1 Code Files}
\label{sec:question-1-code}

\subsection{Source Code}
\label{sec:source-code}

\shcodefile{src/salsa/SalsaParser.hs}{sphs}

\subsection{Test Code}
\label{sec:test-code}

\shcodefile{src/salsa/SalsaParserTest.hs}{spt}

\subsection{Test Output}
\label{sec:test-output}

\begin{lstlisting}[caption=Session output: {\tt src/salsa/SalsaParserTest}, label=lst:testoutparser]
*** Checking Colour ***
pass: circle c 0 0 0 blue
pass: circle c 0 0 0 plum
pass: circle c 0 0 0 red
pass: circle c 0 0 0 green
pass: circle c 0 0 0 orange
error (expected): circle c 0 0 0 violet
error (expected): circle c 0 0 0 greeen
error (expected): circle c 0 0 0 Blue
error (expected): circle c 0 0 0 blue1

*** Checking Prim ***
pass: a -> (0, 0)
pass: a -> (42, 42)
pass: a -> (999999999, 999999999)
pass: a -> ((((42))), (((42))))
pass: a -> (john . x, john . x)
pass: a -> (john . y, john . y)
error (expected): a -> (-5, -5)
error (expected): a -> (-5, -5)
error (expected): a -> (42.2, 42.2)
error (expected): a -> (.8, .8)

*** Checking Expr ***
pass: a -> (1 + 2, 1 + 2)
pass: a -> (1 - 2, 1 - 2)
pass: a -> (1 + 2 + 3, 1 + 2 + 3)
pass: a -> (1 - 2 - 3, 1 - 2 - 3)
pass: a -> (1 + 2 - 3, 1 + 2 - 3)
pass: a -> (1 - 2 + 3, 1 - 2 + 3)

*** Checking Pos ***
pass: a -> (0, 0)
pass: a -> + (0, 0)

*** Checking SIdents ***
pass: a -> (0, 0)
pass: aBC -> (0, 0)
pass: a12T -> (0, 0)
pass: aa bb cc -> (0, 0)
error (expected): 1abc -> (0, 0)
error (expected): Abc -> (0, 0)
error (expected): _abc -> (0, 0)
error (expected): _ -> (0, 0)
error (expected): viewdef -> (0, 0)
error (expected): rectangle -> (0, 0)
error (expected): circle -> (0, 0)
error (expected): view -> (0, 0)
error (expected): group -> (0, 0)
error (expected): blue -> (0, 0)
error (expected): plum -> (0, 0)
error (expected): red -> (0, 0)
error (expected): green -> (0, 0)
error (expected): orange -> (0, 0)

*** Checking VIdents ***
pass: group V [A]
pass: group V [Abc]
pass: group V [A12t]
pass: group V [AA BB CC]
error (expected): group V [1Abc]
error (expected): group V [aBC]
error (expected): group V [_ABC]
error (expected): group V [_]

*** Checking Command ***
pass: a->(0, 0)
pass: a->(0, 0)@V
pass: a->(0, 0)@V@W
pass: a->(0, 0)||b->(0, 0)
pass: a->(0, 0)||b->(0, 0)||c->(0,0)
pass: a->(0, 0)||b->(0, 0)@V
pass: {a->(0, 0)||b->(0, 0)}@V
pass: {{{a->(0, 0)}}}

*** Checking Definition ***
pass: viewdef V 0 0
pass: rectangle r 0 0 0 0 blue
pass: circle c 0 0 0 blue
pass: view V
pass: group G [X Y Z]
error (expected): view1 V

*** Checking Program ***
error (expected): <empty string>
pass: viewdef Default 400 400
rectangle box 10 400 20 20 green
box -> (10, 200)
box -> +(100, 0)
box -> (110,400)
box -> +(0-100, 0)

pass: viewdef One 500 500
viewdef Two 400 400
group Both [One Two]
view Both
rectangle larry 10 350 20 20 blue
rectangle fawn 300 350 15 25 plum

view Two
larry -> (300, 350) || fawn -> (10,350)

view Both
larry fawn -> +(0, 0 - 300)

*** Checking parseFile ***
pass: viewdef One 500 500
viewdef Two 400 400
group Both [One Two]
view Both
rectangle larry 10 350 20 20 blue
rectangle fawn 300 350 15 25 plum

view Two
larry -> (300, 350) || fawn -> (10,350)

view Both
larry fawn -> +(0, 0 - 300)

*** All tests completed successfully ***
\end{lstlisting}


\subsection{Used Hand-outs}
\label{sec:hand-outs-used}

\shcodefile{src/salsa/SalsaAst.hs}{sa}
\shcodefile{src/salsa/SimpleParse.hs}{sp}

\subsection{Sample Files}
\label{sec:sample-files}

\shcodefile{src/salsa/simple.salsa}{simple}

\shcodefile{src/salsa/multi.salsa}{multi}


\clearpage
\section{Question 2 Code Files}
\label{sec:question-2-code}

\subsection{Source Code}
\label{sec:source-2-code}

\shcodefile{src/salsa/SalsaInterp.hs}{si}

\subsection{Test Code}
\label{sec:test-2-code}

\shcodefile{src/salsa/SalsaInterpTest.hs}{sit}

\subsection{Used Hand-outs}
\label{sec:hand-outs-2-used}

See listing \ref{lst:sa}.

\subsection{Sample Files}
\label{sec:sample-2-files}

See listings \ref{lst:simple} and \ref{lst:multi}.

\clearpage
\section{Question 3 Code Files}
\label{sec:question-3-code}

\subsection{Source Code}
\label{sec:source-3-code}

\lstinputlisting[basicstyle=\ttfamily\footnotesize, language={},%
  caption={\tt src/at\_server/at\_server.erl}, label=lst:ats]{../src/at_server/at_server.erl}

\lstinputlisting[basicstyle=\ttfamily\footnotesize, language={},%
  caption={\tt src/at\_server/at\_trans.erl}, label=lst:att]{../src/at_server/at_trans.erl}

\lstinputlisting[basicstyle=\ttfamily\footnotesize, language={},%
  caption={\tt src/at\_server/at\_extapi.erl}, label=lst:ate]{../src/at_server/at_extapi.erl}

\subsection{Test Code}
\label{sec:test-3-code}

\lstinputlisting[basicstyle=\ttfamily\footnotesize, language={},%
  caption={\tt src/at\_server/at\_server\_tests.erl}, label=lst:atst]{../src/at_server/at_server_tests.erl}

\lstinputlisting[basicstyle=\ttfamily\footnotesize, language={},%
  caption={\tt src/at\_server/at\_extapi\_tests.erl}, label=lst:atet]{../src/at_server/at_extapi_tests.erl}

\subsection{Test Output}
\label{sec:test-3-output}

\begin{lstlisting}[caption=Session output: {\tt src/at\_server/at\_server\_tests.erl}, label=lst:testoutserver]
> eunit:test(at_server, [verbose]).
======================== EUnit ========================
module 'at_server'
  module 'at_server_tests'
    at_server_tests: commit_t_competing_test...[0.007 s] ok
    at_server_tests: commit_t_abort_longrunning_test...[0.051 s] ok
    at_server_tests: commit_t_success_test...ok
    at_server_tests: update_t_failure_test...ok
    at_server_tests: update_t_success_test...ok
    at_server_tests: query_t_failure_test...ok
    at_server_tests: begin_t_test...ok
    at_server_tests: query_t_success_test...ok
    at_server_tests: doquery_failure_test...ok
    at_server_tests: doquery_success_test...ok
    [done in 0.087 s]
  [done in 0.087 s]
=======================================================
  All 10 tests passed.
ok
\end{lstlisting}

\begin{lstlisting}[caption=Session output: {\tt src/at\_server/at\_extapi\_tests.erl}, label=lst:testoutextapi]
> eunit:test(at_extapi, [verbose]).
======================== EUnit ========================
module 'at_extapi'
  module 'at_extapi_tests'
    at_extapi_tests: choice_update_test...[0.001 s] ok
    at_extapi_tests: ensure_update_ok_test...ok
    at_extapi_tests: ensure_update_retry_test...[0.202 s] ok
    at_extapi_tests: ensure_update_error_test...ok
    at_extapi_tests: try_update_error_test...ok
    at_extapi_tests: try_update_aborted_test...[0.101 s] ok
    at_extapi_tests: try_update_ok_test...[0.001 s] ok
    at_extapi_tests: abort_test...ok
    [done in 0.328 s]
  [done in 0.328 s]
=======================================================
  All 8 tests passed.
ok
\end{lstlisting}

\begin{lstlisting}[caption=Session output: {\tt
    commit\_t\_competing\_test()} (debug enabled),
  label=lst:testoutextapi, basicstyle=\ttfamily\scriptsize]
at_server_tests.erl:151:<0.3764.0>: Transaction #Ref<0.0.0.31695> aborted
at_server_tests.erl:151:<0.3764.0>: Transaction #Ref<0.0.0.31699> aborted
at_server_tests.erl:151:<0.3764.0>: Transaction #Ref<0.0.0.31704> aborted
at_server_tests.erl:151:<0.3764.0>: Transaction #Ref<0.0.0.31714> aborted
at_server_tests.erl:151:<0.3764.0>: Transaction #Ref<0.0.0.31719> aborted
at_server_tests.erl:151:<0.3764.0>: Transaction #Ref<0.0.0.31723> aborted
at_server_tests.erl:151:<0.3764.0>: Transaction #Ref<0.0.0.31728> aborted
at_server_tests.erl:151:<0.3764.0>: Transaction #Ref<0.0.0.31732> aborted
at_server_tests.erl:151:<0.3764.0>: Transaction #Ref<0.0.0.31736> aborted
at_server_tests.erl:151:<0.3764.0>: Transaction #Ref<0.0.0.31740> aborted
at_server_tests.erl:151:<0.3764.0>: Transaction #Ref<0.0.0.31744> aborted
at_server_tests.erl:151:<0.3764.0>: Transaction #Ref<0.0.0.31748> aborted
at_server_tests.erl:151:<0.3764.0>: Transaction #Ref<0.0.0.31758> aborted
at_server_tests.erl:151:<0.3764.0>: Transaction #Ref<0.0.0.31762> aborted
at_server_tests.erl:151:<0.3764.0>: Transaction #Ref<0.0.0.31768> aborted
at_server_tests.erl:151:<0.3764.0>: Transaction #Ref<0.0.0.31772> aborted
at_server_tests.erl:151:<0.3764.0>: Transaction #Ref<0.0.0.31776> aborted
at_server_tests.erl:151:<0.3764.0>: Transaction #Ref<0.0.0.31780> aborted
at_server_tests.erl:151:<0.3764.0>: Transaction #Ref<0.0.0.31784> aborted
at_server_tests.erl:151:<0.3764.0>: Transaction #Ref<0.0.0.31788> aborted
at_server_tests.erl:151:<0.3764.0>: Transaction #Ref<0.0.0.31790> aborted
at_server_tests.erl:151:<0.3764.0>: Transaction #Ref<0.0.0.31792> aborted
at_server_tests.erl:151:<0.3764.0>: Transaction #Ref<0.0.0.31794> aborted
at_server_tests.erl:151:<0.3764.0>: Transaction #Ref<0.0.0.31796> aborted
at_server_tests.erl:151:<0.3764.0>: Transaction #Ref<0.0.0.31800> aborted
at_server_tests.erl:151:<0.3764.0>: Transaction #Ref<0.0.0.31806> aborted
at_server_tests.erl:151:<0.3764.0>: Transaction #Ref<0.0.0.31810> aborted
at_server_tests.erl:151:<0.3764.0>: Transaction #Ref<0.0.0.31814> aborted
at_server_tests.erl:151:<0.3764.0>: Transaction #Ref<0.0.0.31818> aborted
at_server_tests.erl:151:<0.3764.0>: Transaction #Ref<0.0.0.31822> aborted
at_server_tests.erl:151:<0.3764.0>: Transaction #Ref<0.0.0.31826> aborted
at_server_tests.erl:151:<0.3764.0>: Transaction #Ref<0.0.0.31830> aborted
at_server_tests.erl:151:<0.3764.0>: Transaction #Ref<0.0.0.31832> aborted
at_server_tests.erl:151:<0.3764.0>: Transaction #Ref<0.0.0.31834> aborted
at_server_tests.erl:151:<0.3764.0>: Transaction #Ref<0.0.0.31836> aborted
at_server_tests.erl:151:<0.3764.0>: Transaction #Ref<0.0.0.31840> aborted
at_server_tests.erl:151:<0.3764.0>: Transaction #Ref<0.0.0.31844> aborted
at_server_tests.erl:151:<0.3764.0>: Transaction #Ref<0.0.0.31848> aborted
at_server_tests.erl:151:<0.3764.0>: Transaction #Ref<0.0.0.31852> aborted
at_server_tests.erl:151:<0.3764.0>: Transaction #Ref<0.0.0.31856> aborted
at_server_tests.erl:151:<0.3764.0>: Transaction #Ref<0.0.0.31860> aborted
at_server_tests.erl:151:<0.3764.0>: Transaction #Ref<0.0.0.31864> aborted
at_server_tests.erl:151:<0.3764.0>: Transaction #Ref<0.0.0.31868> aborted
at_server_tests.erl:151:<0.3764.0>: Transaction #Ref<0.0.0.31872> aborted
at_server_tests.erl:151:<0.3764.0>: Transaction #Ref<0.0.0.31874> aborted
at_server_tests.erl:151:<0.3764.0>: Transaction #Ref<0.0.0.31876> aborted
at_server_tests.erl:151:<0.3764.0>: Transaction #Ref<0.0.0.31878> aborted
at_server_tests.erl:151:<0.3764.0>: Transaction #Ref<0.0.0.31882> aborted
at_server_tests.erl:154:<0.3764.0>: ----> Transaction #Ref<0.0.0.31680> committed
at_server_tests.erl:151:<0.3764.0>: Transaction #Ref<0.0.0.31682> aborted
at_server_tests.erl:151:<0.3764.0>: Transaction #Ref<0.0.0.31684> aborted
at_server_tests.erl:151:<0.3764.0>: Transaction #Ref<0.0.0.31686> aborted
at_server_tests.erl:151:<0.3764.0>: Transaction #Ref<0.0.0.31688> aborted
at_server_tests.erl:151:<0.3764.0>: Transaction #Ref<0.0.0.31691> aborted
at_server_tests.erl:151:<0.3764.0>: Transaction #Ref<0.0.0.31693> aborted
at_server_tests.erl:151:<0.3764.0>: Transaction #Ref<0.0.0.31697> aborted
at_server_tests.erl:151:<0.3764.0>: Transaction #Ref<0.0.0.31701> aborted
at_server_tests.erl:151:<0.3764.0>: Transaction #Ref<0.0.0.31706> aborted
at_server_tests.erl:151:<0.3764.0>: Transaction #Ref<0.0.0.31708> aborted
at_server_tests.erl:151:<0.3764.0>: Transaction #Ref<0.0.0.31710> aborted
at_server_tests.erl:151:<0.3764.0>: Transaction #Ref<0.0.0.31712> aborted
at_server_tests.erl:151:<0.3764.0>: Transaction #Ref<0.0.0.31717> aborted
at_server_tests.erl:151:<0.3764.0>: Transaction #Ref<0.0.0.31721> aborted
at_server_tests.erl:151:<0.3764.0>: Transaction #Ref<0.0.0.31725> aborted
at_server_tests.erl:151:<0.3764.0>: Transaction #Ref<0.0.0.31730> aborted
at_server_tests.erl:151:<0.3764.0>: Transaction #Ref<0.0.0.31734> aborted
at_server_tests.erl:151:<0.3764.0>: Transaction #Ref<0.0.0.31738> aborted
at_server_tests.erl:151:<0.3764.0>: Transaction #Ref<0.0.0.31742> aborted
at_server_tests.erl:151:<0.3764.0>: Transaction #Ref<0.0.0.31746> aborted
at_server_tests.erl:151:<0.3764.0>: Transaction #Ref<0.0.0.31750> aborted
at_server_tests.erl:151:<0.3764.0>: Transaction #Ref<0.0.0.31752> aborted
at_server_tests.erl:151:<0.3764.0>: Transaction #Ref<0.0.0.31754> aborted
at_server_tests.erl:151:<0.3764.0>: Transaction #Ref<0.0.0.31756> aborted
at_server_tests.erl:151:<0.3764.0>: Transaction #Ref<0.0.0.31760> aborted
at_server_tests.erl:151:<0.3764.0>: Transaction #Ref<0.0.0.31764> aborted
at_server_tests.erl:151:<0.3764.0>: Transaction #Ref<0.0.0.31766> aborted
at_server_tests.erl:151:<0.3764.0>: Transaction #Ref<0.0.0.31770> aborted
at_server_tests.erl:151:<0.3764.0>: Transaction #Ref<0.0.0.31774> aborted
at_server_tests.erl:151:<0.3764.0>: Transaction #Ref<0.0.0.31778> aborted
at_server_tests.erl:151:<0.3764.0>: Transaction #Ref<0.0.0.31782> aborted
at_server_tests.erl:151:<0.3764.0>: Transaction #Ref<0.0.0.31786> aborted
at_server_tests.erl:151:<0.3764.0>: Transaction #Ref<0.0.0.31798> aborted
at_server_tests.erl:151:<0.3764.0>: Transaction #Ref<0.0.0.31802> aborted
at_server_tests.erl:151:<0.3764.0>: Transaction #Ref<0.0.0.31804> aborted
at_server_tests.erl:151:<0.3764.0>: Transaction #Ref<0.0.0.31808> aborted
at_server_tests.erl:151:<0.3764.0>: Transaction #Ref<0.0.0.31812> aborted
at_server_tests.erl:151:<0.3764.0>: Transaction #Ref<0.0.0.31816> aborted
at_server_tests.erl:151:<0.3764.0>: Transaction #Ref<0.0.0.31820> aborted
at_server_tests.erl:151:<0.3764.0>: Transaction #Ref<0.0.0.31824> aborted
at_server_tests.erl:151:<0.3764.0>: Transaction #Ref<0.0.0.31828> aborted
at_server_tests.erl:151:<0.3764.0>: Transaction #Ref<0.0.0.31838> aborted
at_server_tests.erl:151:<0.3764.0>: Transaction #Ref<0.0.0.31842> aborted
at_server_tests.erl:151:<0.3764.0>: Transaction #Ref<0.0.0.31846> aborted
at_server_tests.erl:151:<0.3764.0>: Transaction #Ref<0.0.0.31850> aborted
at_server_tests.erl:151:<0.3764.0>: Transaction #Ref<0.0.0.31854> aborted
at_server_tests.erl:151:<0.3764.0>: Transaction #Ref<0.0.0.31858> aborted
at_server_tests.erl:151:<0.3764.0>: Transaction #Ref<0.0.0.31862> aborted
at_server_tests.erl:151:<0.3764.0>: Transaction #Ref<0.0.0.31866> aborted
at_server_tests.erl:151:<0.3764.0>: Transaction #Ref<0.0.0.31870> aborted
at_server_tests.erl:151:<0.3764.0>: Transaction #Ref<0.0.0.31880> aborted
\end{lstlisting}

\section{Submitted File Tree}
\label{sec:submitted-files}

\begin{lstlisting}[caption=File tree under {\tt src/}, label=lst:filetree]
STILL TO DO
\end{lstlisting}
\end{document}
